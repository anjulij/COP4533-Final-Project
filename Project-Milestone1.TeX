\documentclass{article}
\usepackage{graphicx} 
\usepackage{amsmath}
\usepackage{hyperref}

\title{Project Milestone 1}
\author{
K. Anjuli Jones (anjulij) 
\\Matthew Dam (3clipseS)
\\Daniel De Oliveira (danherbb)
\\Carlos Vasquez
}
\date{15 February 2026}

\begin{document}

\maketitle

\section*{Project}
GitHub Repository \href{https://github.com/anjulij/COP4533-Final-Project}{Link}
\\Communication method: Discord
\\Member Roles: 
\\ \href{https://docs.google.com/spreadsheets/d/1yOdtYDnJk8dgZALbhoqwJsnvrXZ8V31M7DZmk_DKJYQ/edit?usp=sharing}{Project Gantt Chart}: (including internal deadlines and meetings)

\section*{Milestone 1: Understanding the Problem}
\subsection*{Problem 1}

\noindent \textbf{Step 1.} Input matrix A
$$A = 
\begin{bmatrix}
    12 & 1 & 5 & 3 & 16 \\
    4 & 4 & 13 & 4 & 9 \\
    6 & 8 & 6 & 1 & 2 \\
    14 & 3 & 4 & 8 & 10 \\
\end{bmatrix}
$$

\medskip
\noindent \textbf{Step 2.} Potential profit by selling on each day for each stock.

\medskip
Let $p_{i,j}$ be the potential profit of selling stock $i$ on day $j$. Then,
\begin{itemize}
	\item $p_{1,1} = 0$, $p_{1,2} = 0$, $p_{1,3} = 4$, $p_{1,4} = 2$, $p_{1,5} = 15$
	\item $p_{2,1} = 0$, $p_{2,2} = 0$, $p_{2,3} = 9$, $p_{2,4} = 0$, $p_{2,5} = 5$
	\item $p_{3,1} = 0$, $p_{3,2} = 2$, $p_{3,3} = 0$, $p_{3,4} = 0$, $p_{3,5} = 1$
	\item $p_{4,1} = 0$, $p_{4,2} = 0$, $p_{4,3} = 1$, $p_{4,4} = 5$, $p_{4,5} = 7$
\end{itemize}

\medskip
\noindent \textbf{Step 3.} Identify the day with the highest potential profit for each stock.

\medskip
Based on the calculated values for $p_{i,j}$ in the previous step,
\begin{itemize}
    \item $max_{j}p_{1,j} = 15$, on day $5$
    \item $max_{j}p_{2,j} = 9$, on day $3$
    \item $max_{j}p_{3,j} = 2$, on day $2$
    \item $max_{j}p_{4,j} = 7$, on day $5$
\end{itemize}

\medskip
\noindent \textbf{Step 4.} Determine the stock and day combination that yields the maximum potential profit.

\medskip
Based on step 3, the maximum potential profit is $15$ and is obtained by selling the first stock on the fifth day. The stock would be bought on the second day, therefore the output for this example would be
$$(1, 2, 5, 15)$$

\subsection*{Problem 2}
Work out the given numerical example for Problem-2. That is, You are given a matrix A of dimensions $ \times n$, where each element represents the predicted prices of m different stocks for n consecutive days. Additionally, you are given an integer $k (1 \leq k \leq n)$. Your task is to manually find a sequence of at most $k$ transactions, each involving the purchase and sale of a single stock, that yields the maximum profit. (Refer to problem statement - 2 for output format).

\paragraph{Input Matrix A}

\[
A =
\begin{bmatrix}
    25 & 30 & 15 & 40 & 50 \\
    10 & 20 & 30 & 25 & 5\\
    30 & 45 & 35 & 10 & 15\\
    5 & 50 & 35 & 25 & 45
\end{bmatrix}
\]
\[k = 3\]

\paragraph{Steps}
\begin{enumerate}
    \item  Begin with the input matrix A as provided.
    \item Determine the sequence of at-most K non-overlapping transactions. A valid transaction is a buy-sell of the same stock. Different transactions can have different stocks, but one transaction would deal with only a single stock.
    \item Output should be a sequence of at most K transactions in the format of $(i,j,l)$ that yields the maximum potential profit by selling ith stock on lth day that was bought on jth day.
\end{enumerate}

\subsubsection*{Notes}

\begin{itemize}
  \item This milestone involves manually applying the basic logic of calculating potential profits and selecting the optimal transaction for a single stock.
\item The goal is to understand the foundational logic of finding the maximum profit for atmost K non-overlapping transactions by considering the price differences between different days for a specific
stock (in a single transaction).
\end{itemize}

\subsection*{Problem 3}
\end{document}